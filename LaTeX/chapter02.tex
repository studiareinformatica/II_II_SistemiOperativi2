\section{Linux}
Di seguito i principali tipi di \textit{filesystem} di un sistema \textit{Linux}.
\begin{center}
    \begin{tabular}{|c|c|c|c|c|}
        \hline
        \textbf{Nome} & \textbf{Journal} & \textbf{Partizione (TB)} & \textbf{File (TB)} & \textbf{Filename (bytes)} \\ \hline
        ext2        & N     & 32    & 2     & 255   \\ \hline
        ext3        & Y     & 32    & 2     & 255   \\ \hline
        ext4        & Y     & 1024  & 16    & 255   \\ \hline
        ReiserFS    & Y!    & 16    & 8     & 4032  \\ \hline
    \end{tabular}
\end{center}

\section{Partizionamento}
Per creare una semplicissima partizione, possiamo creare un file di 10MB (10 blocchi da 1MB), composto da soli \textit{0}:
\begin{lstlisting}
dd if=/dev/zero of=file bs=1M count=10
\end{lstlisting}
A questo punto imponiamo al nuovo file \textit{file} un filesystem, ulilizzando il comando \textit{mkfs}:
\begin{lstlisting}
mkfs -t ext4 file
\end{lstlisting}
Possiamo ora montare questo filesystem fisicamente:
\begin{lstlisting}
mount -t ext4 file /punto/di/mount
\end{lstlisting}
Nella directory \textit{/punto/di/mount} avremo la nostra partizione (al momento vuota), appena formattata, di una capienza massima di 100MB.
Qualsiasi file spostato nella directory \textit{/punto/di/mount} sarà persistito e gestito dal filesystem delegato, fino a che il device sarà montato. \\
Per smontare la partizione:
\begin{lstlisting}
umount /punto/di/mount
\end{lstlisting}